\documentclass{beamer}

\usepackage{amsmath}
\usepackage{apacite}

\author{George G Vega\\ \url{mailto:gvegayon@caltech.edu}}
\title{An\'alisis de Redes}
\date{20 de junio, 2014}

\begin{document}

\frame{\maketitle}

\begin{frame}
\frametitle{Contenidos}
\tableofcontents
\end{frame}

\section{Introducci\'on}

\section{Definiciones}

\begin{frame}
\frametitle{Definiciones}

\begin{itemize}
\item Grafo
\item[Vertice] (aka nodo) Unidad fundamental de un grafo.
\item[Conexi\'on] (link) Puede ser dirigida o no.
\item[Grado] (entrada/salida) N\'umero de conexiones de un nodo.
\item[Geod\'esica] Distancia m\'as corta entre dos nodos.
\item[Diametro] Distancia m\'as larga en un g\'rafo. 
\item[Diada] Par de nodos conectados.
\item[Triada] Trio de nodos conectados
\item[Componente] (gigante) Porci\'on de un grafo desconectado.
\item Centralidad*
\item Clusterizacion*
\end{itemize}
\end{frame}

\begin{frame}
\frametitle{Medidas}

M\'aximo n\'umero de conexiones en un grafo
\begin{equation}
\frac{1}{2}n(n-1)
\end{equation}
\end{frame}

\begin{frame}
\frametitle{Mundos Peque\~nos}
item Se puede medir en funci\'on de la geod\'esica media

\begin{equation}
l = \frac{1}{\frac{1}{2}n(n+1)}\sum_{i\geq j}{d_{ij}}
\end{equation}

Notar que $\frac{1}{2}n(n+1)$ corresponde al m\'aximo n\'umero de links
posibles.
\end{frame}

\section{Estad\'isticos}

\begin{frame}
\begin{itemize}
\item Centralidad
  \begin{itemize}
  \item Erdös
  \item Kevin Bacon \url{http://oracleofbacon.org/}
  \end{itemize}
\item Centralidad de inter mediaci\'on
  \begin{itemize}
  \item Eigencentrality
  \end{itemize}
\item Concentraci\'on (cluster)
\end{itemize}
\end{frame}


\begin{frame}

\end{frame}

\section{Ejemplos}

\section{Referencias}
\nocite{*}
\begin{frame}
\frametitle{Referencias}
\bibliographystyle{apacite}
\bibliography{../bib}
\end{frame}


\end{document}

