\documentclass{article}

\usepackage{apacite}
\usepackage{hyperref}
\usepackage[spanish]{babel}
\usepackage[usenames,dvipsnames]{color}
\hypersetup{colorlinks=true,allcolors=blue}

 \renewcommand{\familydefault}{\sfdefault}

\title{{\color{Gray} [borrador]} \\An\'alisis de Redes Sociales\\{\normalsize Software y an\'alisis de datos para las ciencias sociales}}
\author{George G. Vega\thanks{Ingeniero Comercial y Mag\'ister en Econom\'ia y Pol\'iticas Públicas de la Universidad Adolfo Ib\'a\~nez. \href{mailito:g.vegayon@gmail.com}{g.vegayon@gmail.com} \href{http://www.ggvega.com}{http://ggvega.com}}}
\date{vers dic 2013}

\begin{document}

\maketitle

\begin{abstract}
El curso busca introducir al alumno en el an\'alsis de sistemas a trav\'es de la
teor\'ia de grafos. En particular, se revisar\'an los conceptos b\'asicos de la ciencia
de redes en cuanto a propiedades, clases e indicadores, de forma tal de poder entender
los sistemas sociales desde la perspectiva de esta disciplina. El curso finalizará con un taller pr\'actico utilizando el software de visualizaci\'on de redes Gephi.
\end{abstract}

\nocite{*}
\bibliographystyle{apacite}
\bibliography{bib}

\end{document}